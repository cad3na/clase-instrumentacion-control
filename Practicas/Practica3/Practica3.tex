%-------------------------------------------------------------------------------
%	PAQUETES Y OTRAS CONFIGURACIONES
%-------------------------------------------------------------------------------

\input{../style/practica.tex}

%-------------------------------------------------------------------------------
%	TITULO
%-------------------------------------------------------------------------------

\title{
	\normalfont \normalsize
	\begin{figure}[h]
		\begin{center}
			\includegraphics[width=0.3\textwidth]{../images/UNITEC.png} % La imagen esta incluida en el mismo directorio del código
		\end{center}
	\end{figure}
	\textsc{Instrumentación y Control} \\ [25pt]
	\horrule{0.5pt} \\[0.4cm] % Linea horizontal delgada
	\huge Práctica 3 - Medición de variables físicas II \\ % Titulo de la práctica
	\horrule{2pt} \\[0.5cm] % Linea horizontal mas gruesa
}

\author{Roberto Cadena Vega} % Nombre del profesor

\date{\normalsize 12 de junio de 2014} % Fecha de la práctica

%-------------------------------------------------------------------------------
%	EMPIEZA EL DOCUMENTO
%-------------------------------------------------------------------------------

\begin{document}

\maketitle % Imprime el título

%-------------------------------------------------------------------------------
%	OBJETIVOS
%-------------------------------------------------------------------------------

\section{Objetivos}

	Implementar un sistema eléctrico simple, que sea capaz de medir la intensidad luminosa.

%-------------------------------------------------------------------------------
%	CONOCIMIENTOS PREVIOS
%-------------------------------------------------------------------------------

\section{Conocimientos Previos}

%-------------------------------------------------------------------------------

	\subsection{Transductores}

		El día de hoy usaremos un sensor de fotones (es decir, de luz). Estrictamente hablando, este es un transductor solamente, pero podemos utilizarlo para medir una variable física simplemente.

		La fotorresistencia que utilizaremos es básicamente una resistencia, que varía su valor dependiendo de la intensidad luminosa. Por lo general las venden en valores como $10 k \Omega$ o $100 k \Omega$, y esto quiere decir que variarán desde $0 \Omega$, hasta el valor nominal.

		El problema es que la resistencia no es una señal que podamos medir fácilmente, de hecho cuando la medimos con el multímetro, este tiene que aplicar un voltaje de prueba para saber, a través de un cálculo, la resistencia y justo esto es lo que haremos.

%-------------------------------------------------------------------------------

	\subsection{Acondicionamiento}

		En esta ocasión no queremos una señal analógica como salida de nuestro sistema, sino una señal que nos diga simplemente si hay luz o no, por lo que no podemos simplemente amplificar la señal que mediremos en nuestra resistencia.

		\begin{center}
			\tikzstyle{input} = [coordinate]
			\tikzstyle{output} = [coordinate]
			\tikzstyle{block} = [draw, rectangle, minimum height=3em, minimum width=6em]
			\tikzstyle{init} = [pin edge={to-, thin, black}]

			\begin{tikzpicture}[auto, node distance=4cm, >=latex']
				\node [input, name=entrada] {};
				\node [block, right of=entrada] (sensor) {Señal};
				\node [block, right of=sensor] (acondicionador) {Acondicionamiento};
				\node [output, right of=acondicionador] (salida) {};

				\draw [->] (entrada) -- node[name=x] {$x$} (sensor);
				\draw [->] (sensor) -- node[name=x] {$u$} (acondicionador);
				\draw [->] (acondicionador) -- node[name=y] {$y$} (salida);
			\end{tikzpicture}
		\end{center}

		¿Más aún, qué tan oscuro significa que no hay luz? ¿Si alguien saca su celular e ilumina ligeramente el cuarto debemos considerar que hay luz? Tenemos que considerar una referencia.

%-------------------------------------------------------------------------------

		\subsubsection{Comparadores}

			Los OPAMP's que usamos en la práctica anterior tambien pueden hacer una comparación entre dos señales (de hecho, así es como logran amplificar una señal, pero por ahora no entraremos en detalle con eso), por lo que ahora revisaremos cómo se puede configurar un OPAMP para que funcione como comparador.

			\begin{center}
				\begin{circuitikz} \draw
					(0,0) node[op amp](opamp){}
					(opamp.+) to [short, -*] (-1.2, -0.5) node[below] {$V_i$}
					(opamp.-) to [short, -o] (-1.2, 0.5) node[left] {$V_{ref}$}
					(opamp.out) to [short, -o] (1.5, 0) node[right] {$V_o$}
					(opamp.down) node[ground] {}
					(opamp.up) to [short, -o] (-0.08, 1) node[above] {$V_+$}
					(-1.2, -0.5) to [short, *-] (-3, -0.5)
					(-3, -0.5) to [phR, l=$R_F$] (-3, 2.5) node[above] {$5 V$}
					(-3, -0.5) to [R, l=$R_1$] (-3, -3.5) node[ground] {}
				;\end{circuitikz}
			\end{center}

			y la ecuación que rige su comportamiento es:

				\begin{equation}
					V_o = \left\{
					\begin{array}{r l}
						0 V & V_i < V_{ref}  \\
						V_+ & V_i > V_{ref}
					\end{array} \right.
				\end{equation}

			eso quiere decir que el diagrama queda como sigue:

			\begin{center}
				\tikzstyle{input} = [coordinate]
				\tikzstyle{output} = [coordinate]
				\tikzstyle{block} = [draw, rectangle, minimum height=3em, minimum width=6em]
				\tikzstyle{sum} = [draw, circle, node distance=3cm]
				\tikzstyle{init} = [pin edge={to-, thin, black}]
				\tikzstyle{pinstyle} = [pin edge={to-,thin,black}]

				\begin{tikzpicture}[auto, node distance=3cm, >=latex']
					\node [input, name=entrada] {};
					\node [block, right of=entrada] (sensor) {Señal};
					\node [sum, right of=sensor, pin={[pinstyle]above:$- V_{ref}$}] (suma) {};
					\node [block, right of=suma] (amplificador) {$\infty$};
					\node [output, right of=amplificador] (salida) {};

					\draw [->] (entrada) -- node {$x$} (sensor);
					\draw [->] (sensor) -- node {$u$} node[pos=0.99] {$+$} (suma);
					\draw [->] (suma) -- node {$v$} (amplificador);
					\draw [->] (amplificador) -- node {$y$} (salida);
				\end{tikzpicture}
			\end{center}

			Por ahora no te preocupes por el $\infty$, tan solo hay que notar que nuestra salida tendrá un valor de $V_+$ o $0 V$, dependiendo de la entrada.

			Recuerda que dependiendo del OPAMP que uses, las terminales de tu circuito integrado serán diferentes, por lo que es absolutamente necesario que tengas el datasheet de tu OPAMP a la mano.

%-------------------------------------------------------------------------------

	\subsection{Divisor de voltaje}

		Cuando tenemos dos resistencias en serie conectadas a una diferencia de potencial eléctrico (voltaje), decimos que están dividiendo el voltaje, ya que cuando tomamos en cuenta la proporción que guardan, podemos decir el voltaje que se puede medir en el nodo ubicado entre las dos resistencias.

		\begin{center}
			\begin{circuitikz}[american voltages]
				\draw (0, 0) to [V=$5 V$] (0, 2) to [R=$R_1$] (2, 2) to [R=$R_2$] (2, 0) -- (0, 0);
			\end{circuitikz}
		\end{center}

		Como ya te habrás dado cuenta en la práctica pasada, este circuito es equivalente a:

		\begin{center}
			\begin{circuitikz}
				\draw
				(0,2) node[above]{$5 V$} to[R=$R_1$, o-o] (0, 0) to[R=$R_2$,o-o] (0,-2) node[ground]{}
				(0,0) to[short, -o] (1,0) node[right] {$V_ref$}
				;
			\end{circuitikz}
		\end{center}

		Si analizamos por un momento su comportamiento, podremos deducir el $V_{ref}$ ¿Qué hay del voltaje total? ¿Podemos calcularlo?

		\begin{equation}
			V_T = I_T R_T
		\end{equation}

		siendo $R_T = R_1 + R_2$. Pero sabemos que la corriente total es la misma que pasa por todo el circuito (¡está en serie!), por lo que podemos despejarla y usarla como una constante más.

		\begin{equation}
			I_T = \frac{V_T}{R_T} = I_1 = I_2
		\end{equation}

		Si ahora nos fijamos en la resistencia $R_2$ solamente ¿Cuál será el voltaje $V_2$ en esta resistencia?

		\begin{equation}
			V_2 = I_2 R_2
		\end{equation}

		pero sabemos que $I_2 = \frac{V_T}{R_1 + R_2}$, por lo que podemos expresarla asi:

		\begin{equation}
			V_2 = \frac{V_T}{R_1 + R_2} R_2 = \frac{R_2}{R_1 + R_2} V_T
		\end{equation}

		Esta fórmula nos da una manera de diseñar un circuito que transforme un voltaje de alimentación a uno determinado por nosotros, con un cálculo simple y un par de resistencias.

%-------------------------------------------------------------------------------
%	EQUIPO
%-------------------------------------------------------------------------------

\section{Equipo}

	El siguiente equipo será proporcionado por el laboratorio, siempre y cuando lleguen en los primeros 15 minutos de la práctica, y hagan el vale conteniendo el siguiente equipo (exceptuando las pinzas).

	\begin{itemize}
		\item 1 Fuente de Alimentación
		\item 1 Multímetro
		\item 1 Cable de alimentación
		\item 2 Cables banana - caimán
		\item Pinzas
	\end{itemize}

%-------------------------------------------------------------------------------
%	MATERIALES
%-------------------------------------------------------------------------------

\section{Materiales}

	\begin{itemize}
		\item Protoboard
		\item Fotorresistencia de $10 k \Omega$ o $100 k \Omega$ o cualquier otra que consigas
		\item LM358
		\item LED
		\item Zumbador (opcional)
		\item Resistencias (considera una resistencia del mismo valor que la fotorresistencia)
		\begin{itemize}
			\item $180 \Omega$
			\item $220 \Omega$
			\item $330 \Omega$
			\item $1 k\Omega$
			\item $3.3 k\Omega$
			\item $10 k\Omega$
		\end{itemize}
		\item Cables
	\end{itemize}

%-------------------------------------------------------------------------------
%	DESARROLLO
%-------------------------------------------------------------------------------

\section{Desarrollo}

	\begin{enumerate}
		\item Diseña un circuito para medir el voltaje que pasa por la fotorresistencia en serie con otra resistencia (como en la práctica 1, con el LED).
		\item Diseña un circuito para dar un voltaje de referencia al comparador.
		\item Implementa el circuito del OPAMP configurado como comparador.
		\item Realiza las mediciones requeridas en la hoja de anotaciones.
	\end{enumerate}


%-------------------------------------------------------------------------------
%	CONCLUSIONES
%-------------------------------------------------------------------------------

\section{Conclusiones}

	El alumno deberá describir sus conclusiones al final de su reporte de práctica.

%-------------------------------------------------------------------------------
%	CUESTIONARIO
%-------------------------------------------------------------------------------

\clearpage
\section{Cuestionario - Práctica 3}
	Nombre del alumno: \\[0.2cm]
	\horrule{0.5pt} \\[0.2cm] % Linea horizontal delgada

	\begin{enumerate}
		\item Si tenemos las resistencias $R_1 = 180 \Omega$ y $R_2 = 330 \Omega$ ¿Qué voltaje de salida $V_2$ tendremos para una alimentación de $V_T = 5 V$?\\ \\ \\ \\
		\item ¿Qué valores de resistencias $R_1$ y $R_2$ serán necesarias para que con una alimentación de $12 V$ tengamos una salida de $5 V$?\\ \\ \\ \\
		\item Si alimento mi OPAMP con una alimentación de $5 V$ y $0V$ y el voltaje en la entrada inversora es de $2.3 V$ y el voltaje en la entrada inversora es $4 V$ ¿Qué voltaje $V_O$ tendré en la salida de mi amplificador?\\ \\ \\ \\
		\item Si alimento mi OPAMP con una alimentación de $5 V$ y $0V$ y el voltaje en la entrada inversora es de $2.3 V$ y el voltaje en la entrada inversora es $1 V$ ¿Qué voltaje $V_O$ tendré en la salida de mi amplificador?\\ \\ \\ \\
		\item ¿Cuáles resistencias usarías para diseñar el divisor de voltaje que te dé como salida $2.3 V$?\\
	\end{enumerate}

%-------------------------------------------------------------------------------
%	HOJA DE ANOTACIONES
%-------------------------------------------------------------------------------

\clearpage
\section{Hoja de Anotaciones}

	\begin{enumerate}
		\item ¿Cuál es el valor del voltaje medido en la fotorresistencia cuando hay luz sobre ella? \newline
		\item ¿Cuál es el valor del voltaje medido en la fotorresistencia cuando no hay luz sobre ella? \newline
		\item ¿Cuál es el valor del voltaje medido en la fotorresistencia cuando está completamente cubierta? \newline
		\item ¿Qué valor elegirás como $V_{ref}$? \newline
		\item ¿Qué valores de resistencia elegirás para que el divisor de voltaje te dé $V_{ref}$? \newline
		\item Dibuja el diagrama de todo tu sistema. \newline \newline \newline \newline \newline \newline \newline \newline \newline \newline \newline \newline \newline \newline \newline
	\end{enumerate}

	Integrantes del equipo: \\[0.2cm]
	\horrule{0.5pt} \\[0.2cm] % Linea horizontal delgada
	\horrule{0.5pt} \\[0.2cm] % Linea horizontal delgada
	\horrule{0.5pt} \\[0.2cm] % Linea horizontal delgada
	\horrule{0.5pt} % Linea horizontal delgada

	Revisó: \\[0.2cm]
	\horrule{0.5pt} \\% Linea horizontal delgada

%-------------------------------------------------------------------------------
%	FIN DEL DOCUMENTO
%-------------------------------------------------------------------------------

\end{document}
