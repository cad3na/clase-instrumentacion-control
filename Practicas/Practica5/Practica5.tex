%----------------------------------------------------------------------------------------
%	PAQUETES Y OTRAS CONFIGURACIONES
%----------------------------------------------------------------------------------------

\input{../style/practica.tex}

%----------------------------------------------------------------------------------------
%	TITULO
%----------------------------------------------------------------------------------------

\title{
	\normalfont \normalsize
	\begin{figure}[h]
		\begin{center}
			\includegraphics[width=0.3\textwidth]{../images/UNITEC.png} % La imagen esta incluida en el mismo directorio del código
		\end{center}
	\end{figure}
	\textsc{Instrumentación y Control} \\ [25pt]
	\horrule{0.5pt} \\[0.4cm] % Linea horizontal delgada
	\huge Práctica 5 - Actuación II \\ % Titulo de la práctica
	\horrule{2pt} \\[0.5cm] % Linea horizontal mas gruesa
}

\author{Roberto Cadena Vega} % Nombre del profesor

\date{\normalsize 24 de julio de 2014} % Fecha de la práctica

%----------------------------------------------------------------------------------------
%	EMPIEZA EL DOCUMENTO
%----------------------------------------------------------------------------------------

\begin{document}

\maketitle % Imprime el título

%-------------------------------------------------------------------------------
%	OBJETIVOS
%-------------------------------------------------------------------------------

\section{Objetivos}

	Implementar un sistema de control automático de tipo On/Off.

%-------------------------------------------------------------------------------
%	CONOCIMIENTOS PREVIOS
%-------------------------------------------------------------------------------

\section{Conocimientos Previos}

%-------------------------------------------------------------------------------

	\subsection{Control}

		Le llamaremos control, a todas las acciones que podemos tomar, para modificar las variables que queremos regular en nuestro sistema.

		Comunmente el control, lo aplicamos en las señales eléctricas, ya que son el tipo de señales que nos da la mayoria de nuestros sensores, y es el tipo de señal que toman como entrada los elementos de actuación. Digamos que es una manera comoda de manejar nuestras señales.

%-------------------------------------------------------------------------------

	\subsection{Control On/Off}


%-------------------------------------------------------------------------------
%	EQUIPO
%-------------------------------------------------------------------------------

\section{Equipo}

	El siguiente equipo será proporcionado por el laboratorio, siempre y cuando lleguen en los primeros 15 minutos de la práctica, y hagan el vale conteniendo el siguiente equipo (exceptuando las pinzas).

	\begin{itemize}
		\item 1 Fuente de Alimentación
		\item 1 Generador de Funciones
		\item 1 Osciloscopio
		\item 1 Multímetro
		\item 3 Cables de alimentación
		\item 2 Cables banana - caimán
		\item 3 Cables coaxial - caimán
		\item Pinzas
	\end{itemize}

%-------------------------------------------------------------------------------
%	MATERIALES
%-------------------------------------------------------------------------------

\section{Materiales}

	\begin{itemize}
		\item Protoboard
		\item LM35
		\item LM358
		\item 1 ULN2003 (Darlington Array)
		\item 1 Motor CD de $5 V$ - $12 V$ de alimentación
		\item Resistencias
		\begin{itemize}
			\item $180 \Omega$
			\item $220 \Omega$
			\item $330 \Omega$
			\item $470 \Omega$
			\item $1 k\Omega$
			\item $2.2 k\Omega$
			\item $3.3 k\Omega$
			\item $4.7 k\Omega$
			\item $10 k\Omega$
		\end{itemize}
		\item Cables
	\end{itemize}

%-------------------------------------------------------------------------------
%	DESARROLLO
%-------------------------------------------------------------------------------

\section{Desarrollo}

	\begin{enumerate}
		\item Diseña un circuito con tu motor, que se pueda acoplar al generador de funciones, aislandolo de cualquier interacción con el Motor (utiliza un Darlington array para asegurarte de que la corriente tome el motor provenga de la fuente de alimentación y no del generador de funciones).
	\end{enumerate}


%----------------------------------------------------------------------------------------
%	CONCLUSIONES
%----------------------------------------------------------------------------------------

\section{Conclusiones}

	El alumno deberá describir sus conclusiones al final de su reporte de práctica.

%----------------------------------------------------------------------------------------
%	HOJA DE ANOTACIONES
%----------------------------------------------------------------------------------------

\clearpage
\section{Hoja de Anotaciones}

	\begin{enumerate}
		\item Dibuja el diagrama de conexiones para tu circuito. \newline \newline \newline \newline \newline \newline \newline \newline \newline \newline \newline \newline \newline \newline \newline \newline \newline \newline \newline \newline \newline \newline \newline \newline \newline \newline \newline \newline
	\end{enumerate}

	Integrantes del equipo: \\[0.2cm]
	\horrule{0.5pt} \\[0.2cm] % Linea horizontal delgada
	\horrule{0.5pt} \\[0.2cm] % Linea horizontal delgada
	\horrule{0.5pt} \\[0.2cm] % Linea horizontal delgada
	\horrule{0.5pt} % Linea horizontal delgada

	Revisó: \\[0.2cm]
	\horrule{0.5pt} \\% Linea horizontal delgada

%----------------------------------------------------------------------------------------
%	FIN DEL DOCUMENTO
%----------------------------------------------------------------------------------------

\end{document}
