%----------------------------------------------------------------------------------------
%	PAQUETES Y OTRAS CONFIGURACIONES
%----------------------------------------------------------------------------------------

\input{../style/practica.tex}

%----------------------------------------------------------------------------------------
%	TITULO
%----------------------------------------------------------------------------------------

\title{	
	\normalfont \normalsize
	\begin{figure}[h]
		\begin{center}
			\includegraphics[width=0.3\textwidth]{../images/UNITEC.png} % La imagen esta incluida en el mismo directorio del código
		\end{center}
	\end{figure}
	\textsc{Instrumentación y Control} \\ [25pt]
	\horrule{0.5pt} \\[0.4cm] % Linea horizontal delgada
	\huge Práctica 2 - Medición de variables físicas \\ % Titulo de la práctica
	\horrule{2pt} \\[0.5cm] % Linea horizontal mas gruesa
}

\author{Roberto Cadena Vega} % Nombre del profesor

\date{\normalsize 5 de junio de 2014} % Fecha de la práctica

%----------------------------------------------------------------------------------------
%	EMPIEZA EL DOCUMENTO
%----------------------------------------------------------------------------------------

\begin{document}

\maketitle % Imprime el título

%----------------------------------------------------------------------------------------
%	OBJETIVOS
%----------------------------------------------------------------------------------------

\section{Objetivos}

	Implementar un sistema eléctrico simple, que sea capaz de medir la temperatura.

%----------------------------------------------------------------------------------------
%	CONOCIMIENTOS PREVIOS
%----------------------------------------------------------------------------------------

\section{Conocimientos Previos}

%----------------------------------------------------------------------------------------

	\subsection{Sensores}

		Dentro de nuestra materia tenemos distintas tipos de variables que tenemos que controlar para que el proceso se lleve adecuadamente, por lo que necesitamos primero que nada medirlas.

		Los sensores nos ayudarán a hacer esto, pero primero tenemos que aprender a usarlos.

		El sensor que utilizaremos en esta práctica es el LM35, por lo que tendrán que descargar su datasheet, pero tal vez te preguntes qué es un datasheet... Bien, pues todos los elementos electrónicos tienen hojas de especificaciones en los que viene toda la información relevante para implementar su uso en un sistema eléctrico.

		Pues lo primero que tienes que hacer es descargar de internet el datasheet de nuestro sensor, una simple búsqueda en Google de "datasheet LM35", te dará como resultado un archivo en formato PDF que puedes descargar e imprimir, para su uso en la práctica.

		Una vez que tengamos nuestro datasheet, lo que tienes que buscar es el voltaje necesario para alimentar nuestro sensor (en caso de que sea necesario). Otro valor importante es la señal que nos regresará el sensor. Otros factores importantes son el tiempo de respuesta, la resistividad interna del elemento, etc., sin embargo, no serán de utilidad en esta práctica.

	\subsection{Acondicionamiento}

		Como te habrás dado cuenta en tu datasheet, el voltaje que nos entrega como salida el LM35 es de $0 V$ a $500 mV$ si tomamos en cuenta que el rango de temperaturas que queremos medir es de $0 ^oC$ a $50 ^oC$, pero si quisiéramos meter esto en un sistema industrial, normalmente tendríamos que normalizar este rango a $0 V$ a $24 V$.

		¿Cuál sería el número por el que tenemos que multiplicar, para que la señal del LM35 se convierta a la que queremos? Cuando hablamos de señales que tenemos que acondicionar, a este número le llamamos Ganancia.

		\begin{center}
			\tikzstyle{input} = [coordinate]
			\tikzstyle{output} = [coordinate]
			\tikzstyle{block} = [draw, rectangle, minimum height=3em, minimum width=6em]
			\tikzstyle{init} = [pin edge={to-, thin, black}]

			\begin{tikzpicture}[auto, node distance=3cm, >=latex']
				\node [input, name=entrada] {};
				\node [block, right of=entrada] (sensor) {Señal};
				\node [block, right of=sensor] (amplificador) {Ganancia};
				\node [output, right of=amplificador] (salida) {};

				\draw [->] (entrada) -- node[name=x] {$x$} (sensor);
				\draw [->] (sensor) -- node[name=x] {$u$} (amplificador);
				\draw [->] (amplificador) -- node[name=y] {$y$} (salida);
			\end{tikzpicture}
		\end{center}

		Esta ganancia la vamos a obtener con un sistema eléctrico conocido como amplificador.

		\subsubsection{Amplificador}

			Los amplificadores están compuestos de circuitos integrados llamados Amplificadores Operacionales (u OPAMP's para los cuates), resistencias y cables, cosas que ya sabemos usar.

			El diagrama para un amplificador no inversor es el siguiente:

			\begin{center}
				\begin{circuitikz} \draw
					(0,0) node[op amp](opamp){}
					(opamp.+) to [R, l=$R_I$] (-4, -0.5) to [short, -o] (-4, -0.5) node[left] {$V_I$}
					(opamp.-) to [short, *-] (-2, 0.5) to [R, l=$R_I$] (-5, 0.5) node[ground] {}
					(opamp.-) to [short, *-] (-1.2, 2) to [R, l=$R_f$] (1.2, 2)
					(opamp.out) to [short, *-o] (2, 0) node[right] {$V_O$}
					(opamp.out) to [short, *-] (1.2, 2)
					(opamp.down) node[ground] {}
					(opamp.up) to [short, -o] (-0.08, 1) node[above] {$V_+$}
				;\end{circuitikz}	
			\end{center}
			
			y la ecuación que rige su comportamiento es:

				\begin{equation}
					V_O = \left(\frac{R_f}{R_I} + 1\right) V_I
				\end{equation}

			eso quiere decir que el diagrama queda como sigue:

			\begin{center}
				\tikzstyle{input} = [coordinate]
				\tikzstyle{output} = [coordinate]
				\tikzstyle{block} = [draw, rectangle, minimum height=3em, minimum width=6em]
				\tikzstyle{init} = [pin edge={to-, thin, black}]

				\begin{tikzpicture}[auto, node distance=3cm, >=latex']
					\node [input, name=entrada] {};
					\node [block, right of=entrada] (sensor) {Señal};
					\node [block, right of=sensor] (amplificador) {$ \frac{R_f}{R_I} + 1$};
					\node [output, right of=amplificador] (salida) {};

					\draw [->] (entrada) -- node[name=x] {$x$} (sensor);
					\draw [->] (sensor) -- node[name=x] {$u$} (amplificador);
					\draw [->] (amplificador) -- node[name=y] {$y$} (salida);
				\end{tikzpicture}
			\end{center}

			Al ver esto, notamos que tan solo tenemos que calcular la ganancia, y buscar 2 valores de resistencias que nos dé el valor de ganancia deseado.

			Esto se puede ver de la siguiente manera:

			\begin{figure}[h]
		    	\begin{center}
		    		\includegraphics[width=0.75\textwidth]{images/LM35-LM358.png}
		    		\caption{Diagrama eléctrico del circuito a ensamblar.}
		    		\label{dia:elecir}
		    	\end{center}
		    \end{figure}

		    En este ejemplo la ganancia del amplificador es de $11$, esto lo podemos comprobar con la fórmula:

		    \begin{equation}
		    	\frac{R_f}{R_I} + 1 = \frac{10 k \Omega}{1 k \Omega} + 1 = 11
		    \end{equation}

			Recuerda que dependiendo del OPAMP que uses, las terminales de tu circuito integrado serán diferentes, por lo que es absolutamente necesario que tengas el datasheet de tu OPAMP a la mano.

%----------------------------------------------------------------------------------------
%	EQUIPO
%----------------------------------------------------------------------------------------

\section{Equipo}

	El siguiente equipo será proporcionado por el laboratorio, siempre y cuando lleguen en los primeros 15 minutos de la práctica, y hagan el vale conteniendo el siguiente equipo (exceptuando las pinzas).

	\begin{itemize}
		\item 1 Fuente de Alimentación
		\item 1 Multímetro
		\item 1 Cable de alimentación
		\item 2 Cables banana - caimán
		\item Pinzas
	\end{itemize}

%----------------------------------------------------------------------------------------
%	MATERIALES
%----------------------------------------------------------------------------------------

\section{Materiales}

	\begin{itemize}
		\item Protoboard
		\item LM35
		\item LM358
		\item Resistencias
		\begin{itemize}
			\item $180 \Omega$
			\item $220 \Omega$
			\item $330 \Omega$
			\item $1 k\Omega$
			\item $3.3 k\Omega$
			\item $10 k\Omega$
			\item $100 k\Omega$
		\end{itemize}
		\item Cables
	\end{itemize}

%----------------------------------------------------------------------------------------
%	DESARROLLO
%----------------------------------------------------------------------------------------

\section{Desarrollo}
	
	\begin{enumerate}
		\item Diseña un amplificador con la ganancia necesaria para que la salida de nuestro sensor, sea normalizada a un rango de $OV$ a $24V$.
		\item Realiza las mediciones requeridas en la hoja de anotaciones.
	\end{enumerate}

%----------------------------------------------------------------------------------------
%	CONCLUSIONES
%----------------------------------------------------------------------------------------

\section{Conclusiones}

	El alumno deberá describir sus conclusiones al final de su reporte de práctica.
    
%----------------------------------------------------------------------------------------
%	CUESTIONARIO
%----------------------------------------------------------------------------------------

\clearpage
\section{Cuestionario - Práctica 2}
	Nombre del alumno: \\[0.2cm]
	\horrule{0.5pt} \\[0.2cm] % Linea horizontal delgada

	\begin{enumerate}
		\item ¿Cuál es el rango de temperaturas que el sensor LM35 es capaz de medir?\\ \\ \\ \\
		\item ¿Cuántos amplificadores se pueden construir con un solo circuito integrado LM358?\\ \\ \\ \\
		\item ¿Cuál es la ganancia necesaria para que una señal constante de $2 V$ se convierta en una de $10.4 V$?\\ \\ \\ \\
		\item ¿Cuál es la ganancia necesaria para que una señal que varía entre $0 V$ y $2 V$ se convierta en una que varíe de $0 V$ a $8.8 V$?\\ \\ \\ \\
		\item ¿Cuáles resistencias usarías para diseñar ese amplificador (ten en cuenta los valores de resistencias comerciales)?\\
	\end{enumerate}
    
%----------------------------------------------------------------------------------------
%	HOJA DE ANOTACIONES
%----------------------------------------------------------------------------------------

\clearpage
\section{Hoja de Anotaciones}
	
	\begin{enumerate}
		\item ¿Cuál es la ganancia necesaria para normalizar nuestra señal a la requerida? \newline
		\item ¿Cuál es el voltaje de alimentación que necesita el amplificador? \newline
		\item ¿Cuál es el valor de $R_f$ que necesitas para obtener esa ganancia? \newline
		\item ¿Cuál es el valor de $R_I$ que necesitas para obtener esa ganancia? \newline
		\item ¿Cuál es la temperatura ambiente en el momento de tus mediciones? \newline
		\item ¿Cuál es el valor obtenido al medir con el multímetro, el voltaje de salida en el sensor? \newline
		\item ¿Cuál es el valor obtenido al medir con el multímetro, el voltaje de salida en el amplificador? \newline
		\item ¿Qué pasa si acercas una fuente de calor al sensor? \newline
	\end{enumerate}

	Integrantes del equipo: \\[0.2cm]
	\horrule{0.5pt} \\[0.2cm] % Linea horizontal delgada
	\horrule{0.5pt} \\[0.2cm] % Linea horizontal delgada
	\horrule{0.5pt} \\[0.2cm] % Linea horizontal delgada
	\horrule{0.5pt} % Linea horizontal delgada	

	Revisó: \\[0.2cm]
	\horrule{0.5pt} \\% Linea horizontal delgada
    
%----------------------------------------------------------------------------------------
%	FIN DEL DOCUMENTO
%----------------------------------------------------------------------------------------

\end{document}