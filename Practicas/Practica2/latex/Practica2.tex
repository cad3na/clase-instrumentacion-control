%----------------------------------------------------------------------------------------
%	PAQUETES Y OTRAS CONFIGURACIONES
%----------------------------------------------------------------------------------------

\input{../../style/practica.tex}

%----------------------------------------------------------------------------------------
%	TITULO
%----------------------------------------------------------------------------------------

\title{	
	\normalfont \normalsize
	\begin{figure}[h]
		\begin{center}
			\includegraphics[width=0.3\textwidth]{UNITEC.png} % La imagen esta incluida en el mismo directorio del código
		\end{center}
	\end{figure}
	\textsc{Instrumentación y Control} \\ [25pt]
	\horrule{0.5pt} \\[0.4cm] % Linea horizontal delgada
	\huge Práctica 2 - Medición de variables físicas \\ % Titulo de la práctica
	\horrule{2pt} \\[0.5cm] % Linea horizontal mas gruesa
}

\author{Roberto Cadena Vega} % Nombre del profesor

\date{\normalsize 27 de enero de 2014} % Fecha de la práctica

%----------------------------------------------------------------------------------------
%	EMPIEZA EL DOCUMENTO
%----------------------------------------------------------------------------------------

\begin{document}

\maketitle % Imprime el título

%----------------------------------------------------------------------------------------
%	OBJETIVOS
%----------------------------------------------------------------------------------------

\section{Objetivos}

	Implementar un sistema eléctrico simple, que sea capaz de medir la temperatura.

%----------------------------------------------------------------------------------------
%	CONOCIMIENTOS PREVIOS
%----------------------------------------------------------------------------------------

\section{Conocimientos Previos}

%----------------------------------------------------------------------------------------

	\subsection{Sensores}

		Dentro de nuestra materia tenemos distintas tipos de variables que tenemos que controlar para que el proceso se lleve adecuadamente, por lo que necesitamos primero que nada medirlas.

		Los sensores nos ayudarán a hacer esto, pero primero tenemos que aprender a usarlos.

		El sensor que utilizaremos en esta práctica es el LM35, por lo que tendrán que descargar su datasheet, pero tal vez te preguntes que es un datasheet... Bien, pues todos los elementos electrónicos tienen hojas de especificaciones en los que viene toda la información relevante para implementar su uso en un sistema eléctrico.

		Pues lo primero que tienes que hacer es descargar de internet el datasheet de nuestro sensor, una simple búsqueda en Google de "datasheet LM35", te dará como resultado un archivo en formato PDF que puedes descargar e imprimir, para su uso en la práctica.

		Una vez que tengamos nuestro datasheet, lo que tienes que buscar es el voltaje necesario para alimentar nuestro sensor (en caso de que sea necesario). Otro valor importante es la señal que nos regresará el sensor. Otros factores importantes son el tiempo de respuesta, la resistividad interna del elemento, etc., sin embargo, no serán de utilidad en esta práctica.

	\subsection{Acondicionamiento}

		Como te habrás dado cuenta en tu datasheet, el voltaje que nos entrega como salida el LM35 es de $0 V$ a $5 V$ si tomamos en cuenta que el rango de temperaturas que queremos medir es de $0 ^oC$ a $50 ^oC$, pero si quisiéramos meter esto en un sistema industrial, normalmente tendríamos que normalizar este rango a $0 V$ a $24 V$.

		Cual sería el numero por el que tenemos que multiplicar, para que a señal del LM35 se convierta a la que queremos? Cuando hablamos de señales que tenemos que acondicionar, este numero le llamamos Ganancia.

		\begin{center}
			\tikzstyle{input} = [coordinate]
			\tikzstyle{output} = [coordinate]
			\tikzstyle{block} = [draw, rectangle, minimum height=3em, minimum width=6em]
			\tikzstyle{init} = [pin edge={to-, thin, black}]

			\begin{tikzpicture}[auto, node distance=3cm, >=latex']
				\node [input, name=entrada] {};
				\node [block, right of=entrada] (sensor) {Señal};
				\node [block, right of=sensor] (amplificador) {Ganancia};
				\node [output, right of=amplificador] (salida) {};

				\draw [->] (entrada) -- node[name=x] {$x$} (sensor);
				\draw [->] (sensor) -- node[name=x] {$u$} (amplificador);
				\draw [->] (amplificador) -- node[name=y] {$y$} (salida);
			\end{tikzpicture}
		\end{center}

%----------------------------------------------------------------------------------------
%	EQUIPO
%----------------------------------------------------------------------------------------

\section{Equipo}

	El siguiente equipo será proporcionado por el laboratorio, siempre y cuando lleguen en los primeros 15 minutos de la práctica, y hagan el vale conteniendo el siguiente equipo (exceptuando las pinzas).

	\begin{itemize}
		\item 1 Fuente de Alimentación
		\item 1 Multimetro
		\item 1 Cable de alimentación
		\item 2 Cables banana - caimán
		\item Pinzas
	\end{itemize}

%----------------------------------------------------------------------------------------
%	MATERIALES
%----------------------------------------------------------------------------------------

\section{Materiales}

	\begin{itemize}
		\item Protoboard
		\item LM35
		\item LM741 o LM358
		\item Resistencias
		\begin{itemize}
			\item $220 \Omega$
			\item $330 \Omega$
			\item $1 k\Omega$
			\item $3.3 k\Omega$
			\item $10 k\Omega$
		\end{itemize}
		\item Cables
	\end{itemize}

%----------------------------------------------------------------------------------------
%	DESARROLLO
%----------------------------------------------------------------------------------------

\section{Desarrollo}


%----------------------------------------------------------------------------------------
%	CONCLUSIONES
%----------------------------------------------------------------------------------------

\section{Conclusiones}

	El alumno deberá describir sus conclusiones al final de su reporte de práctica.
    
%----------------------------------------------------------------------------------------
%	HOJA DE ANOTACIONES
%----------------------------------------------------------------------------------------

\clearpage
\section{Hoja de Anotaciones}

	Anota los pasos a seguir para utilizar correctamente la fuente de alimentación.

	\begin{enumerate}
		\item
		\item
		\item
		\item
		\item
		\item
	\end{enumerate}

	Anota los pasos a seguir para utilizar correctamente el multimetro como Voltmetro.

	\begin{enumerate}
		\item
		\item
		\item
	\end{enumerate}

	Realiza las mediciones de voltaje y calcula la corriente para la resistencia en el circuito:

	\begin{center}
		\begin{tabular}{|p{1.5cm}|p{1.5cm}|p{1.5cm}|}
			\hline
			$V$ & $I$ & $R$          \\
			\hline
			    &     & $220 \Omega$ \\
			\hline
			    &     & $330 \Omega$ \\
			\hline
			    &     & $1 k \Omega$ \\
			\hline
		\end{tabular}
	\end{center}

	Realiza las mediciones de voltaje y calcula la resistencia del LED en el circuito (toma en cuenta que la corriente en el LED, es la misma que en la resistencia, debido a que están conectadas en serie):

	\begin{center}
		\begin{tabular}{|p{1.5cm}|p{1.5cm}|p{1.5cm}|}
			\hline
			$V$ & $I$ & $R$ \\
			\hline
			    &     &     \\
			\hline
			    &     &     \\
			\hline
			    &     &     \\
			\hline
		\end{tabular}
	\end{center}

	Integrantes del equipo: \\[0.2cm]
	\horrule{0.5pt} \\[0.2cm] % Linea horizontal delgada
	\horrule{0.5pt} \\[0.2cm] % Linea horizontal delgada
	\horrule{0.5pt} % Linea horizontal delgada	

	Revisó: \\[0.2cm]
	\horrule{0.5pt} \\% Linea horizontal delgada
    
%----------------------------------------------------------------------------------------
%	FIN DEL DOCUMENTO
%----------------------------------------------------------------------------------------

\end{document}