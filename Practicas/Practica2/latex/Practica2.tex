%----------------------------------------------------------------------------------------
%	PAQUETES Y OTRAS CONFIGURACIONES
%----------------------------------------------------------------------------------------

\input{../../style/practica.tex}

%----------------------------------------------------------------------------------------
%	TITULO
%----------------------------------------------------------------------------------------

\title{	
	\normalfont \normalsize
	\begin{figure}[h]
		\begin{center}
			\includegraphics[width=0.3\textwidth]{UNITEC.png} % La imagen esta incluida en el mismo directorio del código
		\end{center}
	\end{figure}
	\textsc{Instrumentación y Control} \\ [25pt]
	\horrule{0.5pt} \\[0.4cm] % Linea horizontal delgada
	\huge Práctica 2 - Medición de variables físicas \\ % Titulo de la práctica
	\horrule{2pt} \\[0.5cm] % Linea horizontal mas gruesa
}

\author{Roberto Cadena Vega} % Nombre del profesor

\date{\normalsize 27 de enero de 2014} % Fecha de la práctica

%----------------------------------------------------------------------------------------
%	EMPIEZA EL DOCUMENTO
%----------------------------------------------------------------------------------------

\begin{document}

\maketitle % Imprime el título

%----------------------------------------------------------------------------------------
%	OBJETIVOS
%----------------------------------------------------------------------------------------

\section{Objetivos}

	Implementar un sistema eléctrico simple, que sea capaz de medir la temperatura.

%----------------------------------------------------------------------------------------
%	CONOCIMIENTOS PREVIOS
%----------------------------------------------------------------------------------------

\section{Conocimientos Previos}

%----------------------------------------------------------------------------------------

	\subsection{Sensores}

		Dentro de nuestra materia tenemos distintas tipos de variables que tenemos que controlar para que el proceso se lleve adecuadamente, por lo que necesitamos primero que nada medirlas.

		Los sensores nos ayudarán a hacer esto, pero primero tenemos que aprender a usarlos.



%----------------------------------------------------------------------------------------
%	EQUIPO
%----------------------------------------------------------------------------------------

\section{Equipo}

	El siguiente equipo será proporcionado por el laboratorio, siempre y cuando lleguen en los primeros 15 minutos de la práctica, y hagan el vale conteniendo el siguiente equipo (exceptuando las pinzas).

	\begin{itemize}
		\item 1 Fuente de Alimentación
		\item 1 Multimetro
		\item 1 Cable de alimentación
		\item 2 Cables banana - caimán
		\item Pinzas
	\end{itemize}

%----------------------------------------------------------------------------------------
%	MATERIALES
%----------------------------------------------------------------------------------------

\section{Materiales}

	\begin{itemize}
		\item Protoboard
		\item LED (no importa el color, aunque los difusos son mas fáciles de ver en las condiciones de iluminación del laboratorio)
		\item LM35
		\item LM741
		\item Resistencias
		\begin{itemize}
			\item $220 \Omega$
			\item $330 \Omega$
			\item $1 k\Omega$
		\end{itemize}
		\item Cables
	\end{itemize}

%----------------------------------------------------------------------------------------
%	DESARROLLO
%----------------------------------------------------------------------------------------

\section{Desarrollo}


%----------------------------------------------------------------------------------------
%	CONCLUSIONES
%----------------------------------------------------------------------------------------

\section{Conclusiones}

	El alumno deberá describir sus conclusiones al final de su reporte de práctica.
    
%----------------------------------------------------------------------------------------
%	HOJA DE ANOTACIONES
%----------------------------------------------------------------------------------------

\clearpage
\section{Hoja de Anotaciones}

	Anota los pasos a seguir para utilizar correctamente la fuente de alimentación.

	\begin{enumerate}
		\item
		\item
		\item
		\item
		\item
		\item
	\end{enumerate}

	Anota los pasos a seguir para utilizar correctamente el multimetro como Voltmetro.

	\begin{enumerate}
		\item
		\item
		\item
	\end{enumerate}

	Realiza las mediciones de voltaje y calcula la corriente para la resistencia en el circuito:

	\begin{center}
		\begin{tabular}{|p{1.5cm}|p{1.5cm}|p{1.5cm}|}
			\hline
			$V$ & $I$ & $R$          \\
			\hline
			    &     & $220 \Omega$ \\
			\hline
			    &     & $330 \Omega$ \\
			\hline
			    &     & $1 k \Omega$ \\
			\hline
		\end{tabular}
	\end{center}

	Realiza las mediciones de voltaje y calcula la resistencia del LED en el circuito (toma en cuenta que la corriente en el LED, es la misma que en la resistencia, debido a que están conectadas en serie):

	\begin{center}
		\begin{tabular}{|p{1.5cm}|p{1.5cm}|p{1.5cm}|}
			\hline
			$V$ & $I$ & $R$ \\
			\hline
			    &     &     \\
			\hline
			    &     &     \\
			\hline
			    &     &     \\
			\hline
		\end{tabular}
	\end{center}

	Integrantes del equipo: \\[0.2cm]
	\horrule{0.5pt} \\[0.2cm] % Linea horizontal delgada
	\horrule{0.5pt} \\[0.2cm] % Linea horizontal delgada
	\horrule{0.5pt} % Linea horizontal delgada	

	Revisó: \\[0.2cm]
	\horrule{0.5pt} \\% Linea horizontal delgada
    
%----------------------------------------------------------------------------------------
%	FIN DEL DOCUMENTO
%----------------------------------------------------------------------------------------

\end{document}