%-------------------------------------------------------------------------------
%	PAQUETES Y OTRAS CONFIGURACIONES
%-------------------------------------------------------------------------------

\input{../style/practica.tex}

%-------------------------------------------------------------------------------
%	TITULO
%-------------------------------------------------------------------------------

\title{
	\normalfont \normalsize
	\begin{figure}[h]
		\centering \includegraphics[width=0.3\textwidth]{../images/UNITEC.png}
	\end{figure}
	\textsc{Instrumentación y Control} \\ [25pt]
	\horrule{0.5pt} \\[0.4cm] % Linea horizontal delgada
	\huge Práctica 4 - Actuación I \\ % Titulo de la práctica
	\horrule{2pt} \\[0.5cm] % Linea horizontal mas gruesa
}

\author{Roberto Cadena Vega} % Nombre del profesor

\date{\normalsize 18 de julio de 2014} % Fecha de la práctica

%-------------------------------------------------------------------------------
%	EMPIEZA EL DOCUMENTO
%-------------------------------------------------------------------------------

\begin{document}

\maketitle % Imprime el título

%-------------------------------------------------------------------------------
%	OBJETIVOS
%-------------------------------------------------------------------------------

\section{Objetivos}

	Implementar un sistema eléctrico simple, que sea capaz de controlar el brillo de un LED.

%-------------------------------------------------------------------------------
%	CONOCIMIENTOS PREVIOS
%-------------------------------------------------------------------------------

\section{Conocimientos Previos}

%-------------------------------------------------------------------------------

	\subsection{Control}

		Le llamaremos control, a todas las acciones que podemos tomar, para modificar las variables que queremos regular en nuestro sistema.

		Comunmente el control, lo aplicamos en las señales eléctricas, ya que son el tipo de señales que nos da la mayoria de nuestros sensores, y es el tipo de señal que toman como entrada los elementos de actuación. Digamos que es una manera comoda de manejar nuestras señales.

%-------------------------------------------------------------------------------

	\subsection{Señal Eléctrica}

		Una señal eléctrica tiene varias caracteristicas importantes, de las cuales depende en gran medida la acción de salida. El voltaje aplicado es uno de los parametros mas importantes, mientras mas voltaje, mayor la velocidad con la que girará un motor, mayor el brillo de un LED, etc.

		Otro de los parametros importantes es la corriente, y esta es la que completa la potencia de un elemento; es decir, que el voltaje $V$, por la corriente $I$ es igual a la potencia de un elemento eléctrico.

		\begin{equation}
			P = V I
		\end{equation}

		Y en general, obtenemos que la potencia de un elemento eléctrico es constante. Es decir, que mientras mas voltaje le metamos, menos corriente consumirá, y mientras menos voltaje le demos, mas corriente consumirá. Por lo general, no es buena idea dejar que los elementos eléctricos tomen mas corriente de la que deben, ya que como el voltaje suminsitrado es constante, si empieza a consumir más corriente, sube la potencia de trabajo del elemento, y se quema.

		Pues resulta que tenemos que controlar el voltaje dado a un motor eléctrico ¿Como podemos variar el voltaje suministrado a un elemento eléctrico?

		La manera mas facil de hacerlo, es lo que realizaste para dar un voltaje de referencia en la práctica pasada, eso se llama divisor de voltaje.

%-------------------------------------------------------------------------------

	\subsection{Divisor de voltaje}

		Cuando tenemos dos resistencias en serie conectadas a una diferencia de potencial eléctrico (voltaje), decimos que estan dividiendo el voltaje, ya que cuando tomamos en cuenta la proporción que guardan, podemos decir el voltaje que se puede medir en el nodo ubicado entre las dos resistencias.

		\begin{center}
			\begin{circuitikz}[american voltages]
				\draw (0, 0) to [V=$5 V$] (0, 2) to [R=$R_1$] (2, 2) to [R=$R_2$] (2, 0) -- (0, 0);
			\end{circuitikz}
		\end{center}

		Como ya te habrás dado cuenta en la práctica pasada, este circuito es equivalente a:

		\begin{center}
			\begin{circuitikz}
				\draw
				(0,2) node[above]{$5 V$} to[R=$R_1$, o-o] (0, 0) to[R=$R_2$,o-o] (0,-2) node[ground]{}
				(0,0) to[short, -o] (1,0) node[right] {$V_ref$}
				;
			\end{circuitikz}
		\end{center}

		Si analizamos por un momento su comportamiento, podremos deducir el $V_{ref}$ ¿Que hay del voltaje total? ¿Podemos calcularlo?

		\begin{equation}
			V_T = I_T R_T
		\end{equation}

		siendo $R_T = R_1 + R_2$. Pero sabemos que la corriente total es la misma que pasa por todo el circuito (esta en serie!), por lo que podemos despejarla y usarla como una constante mas.

		\begin{equation}
			I_T = \frac{V_T}{R_T} = I_1 = I_2
		\end{equation}

		Si ahora nos fijamos en la resistencia $R_2$ solamente ¿Cual sera el voltaje $V_2$ en esta resistencia?

		\begin{equation}
			V_2 = I_2 R_2
		\end{equation}

		pero sabemos que $I_2 = \frac{V_T}{R_1 + R_2}$, por lo que podemos expresarla asi:

		\begin{equation}
			V_2 = \frac{V_T}{R_1 + R_2} R_2 = \frac{R_2}{R_1 + R_2} V_T
		\end{equation}

		Esta manera de controlar el voltaje no es la óptima para controlar elementos eléctricos como los motores, debido a la baja impedancia de los motores, y su alto consumo de corriente eléctrica. A continuación veremos una manera mas de controlar el voltaje.

%----------------------------------------------------------------------------------------

	\subsection{PWM}

		Otra manera de controlar el voltaje suministrado a un elemento de actuación, es utilizar un PWM\footnote{Pulse Width Modulation (Modulación por Ancho de Pulso).}. Básicamente, lo que tratamos de hacer es apagar y prender rapidamente el elemento eléctrico a controlar, de manera que el voltaje que alcanze a ver el elemento eléctrico no sea completamente el que le suministremos, sino uno proporcional al tiempo en que lo dejamos prendido.

		Veamos una gráfica:



		Como puedes ver le estamos suministrando un pulso de $5 V$, pero no es constante, digamos que esta prendido el $50\%$ del tiempo. En dado caso, decimos que el $DC$\footnote{Duty Cycle (Ciclo de Trabajo).} es de $50\%$, y la fórmula para el voltaje aparente en el elemento eléctrico conectado es:

		\begin{equation}
			V_{PWM} = V_{pico} \cdot DC
		\end{equation}

		Así pues, en nuestro ejemplo, el voltaje sería:

		\begin{equation}
			V_{PWM} = 5 V \cdot 0.50 = 2.5 V
		\end{equation}

		Este tipo de señal es cuadrada, y a parte del $DC$, tiene un parametro muy importante, la frecuencia $f$. Por lo general utilizamos $f$, que esten por encima de los $20 kHz$, debido a que el sistema eléctrico va a oscilar, y estas oscilaciones pudieran ser audibles, dependiendo de los elementos eléctrico involucrados, y el oido humano solo es capaz de oir frecuencias entre los $20 Hz$ y $20 kHz$. Sin embargo, tambien tenemos que considerar los tiempos de respuesta de los elementos eléctricos mas complejos, por lo que pudiera ser necesario utilizar una frecuencia menor.

		Esta señal cuadrada puede ser generada por el generador de funciones, y debes estar atento para cuando se describa su funcionamiento en el momento de la práctica.

%----------------------------------------------------------------------------------------

	\subsection{LED}

		Los LED's son elementos eléctricos con una resistencia\footnote{Un LED no tiene resistencia, como pudimos confirmar en la práctica 1, pero sabemos que gasta energía, y por lo tanto, al estar conectado al circuito tiene propiedades parecidas a las de una resistencia.} muy parecida a la de las resistencias comerciales, y con una potencia de trabajo muy baja, por lo que son ideales para controlarlos a traves de resistencias.

		Podemos hacer un divisor de voltaje, que alimente un LED y por lo tanto varie el voltaje suministrado al LED; cambiando asi la intensidad luminosa del LED.

		\begin{center}
			\begin{circuitikz}
				\draw
				(0,2) node[above]{$5 V$} to[R=$R_1$, o-o] (0, 0) to[R=$R_2$,o-o] (0,-2) node[ground]{}
				(0,0) to[short] (1,0) to[led] (1,-2) node[ground]{}
				;
			\end{circuitikz}
		\end{center}

		Este divisor de voltaje es constante, debido a que las resistencias no se pueden variar, pero conocemos un elemento resistivo variable. El potenciometro.

%----------------------------------------------------------------------------------------

	\subsection{Potenciometro}
		Un potenciometro es una resistencia que tiene una sonda a la que se le puede modificar su posición. Cuando se modifica su posición, puede dar una resistencia menor al maximo para el que esta construido, asi pues si el potenciometro esta marcado como de $10 k \Omega$, la mayor resistencia que te podrá dar es de $10 k \Omega$ y la menor de $0 \Omega$.

		Basicamente un potenciometro es un divisor de voltaje, construido de tal manera que las dos resistencias dependan de la posición de la sonda.

		\begin{center}
			\begin{circuitikz}
				\draw
				(0,2) node[above]{} to[R=$R_1$, o-o] (0, 0) to[R=$R_2$,o-o] (0,-2) node[below]{}
				(0,0) to[short, -o] (1,0) node[right] {}
				;
			\end{circuitikz}
		\end{center}

		Si el dial del potenciometro esta al 50\% del final, $R_1 = R_2$, si esta al 66.6\%, $R_1 = 2 R_2$, y asi.

%----------------------------------------------------------------------------------------
%	EQUIPO
%----------------------------------------------------------------------------------------

\section{Equipo}

	El siguiente equipo será proporcionado por el laboratorio, siempre y cuando lleguen en los primeros 15 minutos de la práctica, y hagan el vale conteniendo el siguiente equipo (exceptuando las pinzas).

	\begin{itemize}
		\item 1 Fuente de Alimentación
		\item 1 Generador de Funciones
		\item 1 Osciloscopio
		\item 1 Multímetro
		\item 1 Cable de alimentación
		\item 2 Cables banana - caimán
		\item 3 Cables coaxial - caimán
		\item Pinzas
	\end{itemize}

%----------------------------------------------------------------------------------------
%	MATERIALES
%----------------------------------------------------------------------------------------

\section{Materiales}

	\begin{itemize}
		\item Protoboard
		\item 1 LED
		\item Potenciometro de $10 k \Omega$ (o cualquier otro que consigas)
		\item Resistencias
		\begin{itemize}
			\item $180 \Omega$
			\item $220 \Omega$
			\item $330 \Omega$
			\item $1 k\Omega$
			\item $3.3 k\Omega$
			\item $10 k\Omega$
		\end{itemize}
		\item Cables
	\end{itemize}

%----------------------------------------------------------------------------------------
%	DESARROLLO
%----------------------------------------------------------------------------------------

\section{Desarrollo}

	\begin{enumerate}
		\item Diseña divisores de voltaje para suministrar una alimentación de $2 V$, $2.5 V$ y $3 V$. Toma como voltaje de alimentación general $5 V$.
		\item Diseña un divisor de voltaje variable con un potenciometro. Recuerda que al LED no le puedes suministrar mas de $3 V$ sin que se queme.
		\item Diseña una señal de PWM que nos suministre voltajes aparentes de $2 V$, $2.5 V$ y $3 V$. Toma como voltaje pico $5 V$.
	\end{enumerate}


%----------------------------------------------------------------------------------------
%	CONCLUSIONES
%----------------------------------------------------------------------------------------

\section{Conclusiones}

	El alumno deberá describir sus conclusiones al final de su reporte de práctica.

%----------------------------------------------------------------------------------------
%	HOJA DE ANOTACIONES
%----------------------------------------------------------------------------------------

\clearpage
\section{Hoja de Anotaciones}

	\begin{enumerate}
		\item ¿Cuales son los valores de las resistencias utilizadas para el divisor de voltaje de $2 V$? \newline
		\item ¿Cuales son los valores de las resistencias utilizadas para el divisor de voltaje de $2.5 V$? \newline
		\item ¿Cuales son los valores de las resistencias utilizadas para el divisor de voltaje de $3 V$? \newline
		\item Si quiero que el divisor de voltaje variable me suministre $2 V$ y esta conectado a una diferencia de potencial de $5 V$ ¿A que porcentaje de la posición final lo obtendré? \newline
		\item Si quiero que el divisor de voltaje variable me suministre $2.5 V$ y esta conectado a una diferencia de potencial de $5 V$ ¿A que porcentaje de la posición final lo obtendré? \newline
		\item Si quiero que el divisor de voltaje variable me suministre $3 V$ y esta conectado a una diferencia de potencial de $5 V$ ¿A que porcentaje de la posición final lo obtendré? \newline
		\item Si tengo una señal PWM de $5 V$ de $V_{pico}$ ¿Cual será el $DC$ que me de un voltaje aparente de $2.2 V$?\newline
		\item Dibuja los diagramas para todos los sistemas que diseñaste en la parte posterior de esta hoja (Divisores de voltaje fijos a $2 V$, $2.5 V$ y $3 V$, divisor de voltaje variable, conexion con generador de funciones). \newline \newline \newline \newline \newline
	\end{enumerate}

	Integrantes del equipo: \\[0.2cm]
	\horrule{0.5pt} \\[0.2cm] % Linea horizontal delgada
	\horrule{0.5pt} \\[0.2cm] % Linea horizontal delgada
	\horrule{0.5pt} \\[0.2cm] % Linea horizontal delgada
	\horrule{0.5pt} % Linea horizontal delgada

	Revisó: \\[0.2cm]
	\horrule{0.5pt} \\% Linea horizontal delgada

%----------------------------------------------------------------------------------------
%	FIN DEL DOCUMENTO
%----------------------------------------------------------------------------------------

\end{document}
